\documentclass{article}
\usepackage[head=13pt,%
paperwidth=8.5in, paperheight=11in,
includeheadfoot=false,
% top=1in, bottom=1in, inner=0.75in, outer=0.75in,
% marginparwidth=2pc,
heightrounded
]{geometry}

\usepackage{amssymb}
\usepackage{amsmath}
\usepackage{amsthm}
\usepackage[dvipsnames]{xcolor}
\usepackage{bbm}
\usepackage{hyperref}
\usepackage{listings}
\usepackage{ebproof}
\usepackage{stmaryrd}
\usepackage[square,numbers]{natbib}
\usepackage{titlesec}

\titleformat{\section}
{\normalfont\Large\bfseries}{Question~\thesection}{1em}{}

\lstdefinelanguage{GCIC}{
  basicstyle=\color{OliveGreen}\ttfamily,
  breaklines=true,
  breakatwhitespace=true,
}
\lstdefinelanguage{CCIC}{
  basicstyle=\color{BlueViolet}\ttfamily,
  breaklines=true,
  breakatwhitespace=true,
}

\newtheorem{theorem}{Theorem}
\theoremstyle{definition}
\newtheorem{example}[theorem]{Example}

\newcommand{\Scode}[1]{{\fontfamily{cmss}\selectfont\color{Mahogany}#1}}
\newcommand{\Gcode}[1]{{\color{OliveGreen}\textit{#1}}}
\newcommand{\Ccode}[1]{{\color{BlueViolet}\textbf{#1}}}
\newcommand{\Scmath}[1]{{\color{Mahogany}#1}}
\newcommand{\Gcmath}[1]{{\color{OliveGreen}#1}}
\newcommand{\Ccmath}[1]{{\color{BlueViolet}\boldsymbol{#1}}}
\newcommand{\Gcsub}[2]{\Gcode{#1}\(\Gcmath{_{#2}}\)}
\newcommand{\GCICN}[0]{GCIC\(^\mathcal{N}\)}
\newcommand{\GCICG}[0]{GCIC\(^\mathcal{G}\)}
\newcommand{\GCICS}[0]{GCIC\(^\uparrow\)}
\newcommand{\CCICN}[0]{CastCIC\(^\mathcal{N}\)}
\newcommand{\CCICG}[0]{CastCIC\(^\mathcal{G}\)}
\newcommand{\CCICS}[0]{CastCIC\(^\uparrow\)}
\newcommand{\GGEq}[0]{\(\triangleright\)GEq}
\newcommand{\ElabSynth}[4]{
  #1~\(\vdash\)~#2~\(\rightsquigarrow\)~#3~\(\triangleright\)~#4}
\newcommand{\ElabCheck}[4]{#1~\(\vdash\)~#2~\(\triangleleft\)~#3~\(\rightsquigarrow\)~#4}
\newcommand{\ElabConstrSynth}[5]{
  #1~\(\vdash\)~#2~\(\rightsquigarrow\)~#3~\(\blacktriangleright_{#4}\)~#5}
\newcommand{\Cast}[3]{
  \(\Ccmath{\langle}\)#1~\(\Ccmath{\Leftarrow}\)~#2\(\Ccmath{\rangle}\)~#3}
\newcommand{\CastTwo}[4]{
  \(\Ccmath{\langle}\)#1~\(\Ccmath{\Leftarrow}\)~#2~\(\Ccmath{\Leftarrow}\)~#3\(\Ccmath{\rangle}\)~#4}
\newcommand{\Rfa}[0]{\textit{refocus}_\textit{aux}}
\newcommand{\Spine}[1]{\Ccmath{\llparenthesis#1\rrparenthesis}}

\title{Qualifying Exam: Towards Implementing a Gradual Dependently Typed
  Language for Programming}

\author{Darshal Shetty}

\date{}

\begin{document}

\maketitle

\section{Literature Review}%
\label{sec:question1}

\subsection{The Question}
What are the main scientific results and problems that have been solved already
in the literature on gradual typing with dependent types? Identify the points in
the design space that have been explored and what properties they have. Are
there any tensions or tradeoffs that prevent the designs from having all the
desirable properties? In your literature search, make sure to include at least
the following papers:

\begin{enumerate}
  \item Approximate Normalization for Gradual Dependent Types, ICFP
    2019.\cite{eremondi_approximate_2019}
  \item On The Design of a Gradual Dependently Typed Language for Programming,
    PhD Thesis 2023.\cite{eremondi_design_2023}
  \item Gradualizing the Calculus of Inductive Constructions, TOPLAS
    2022.\cite{lennon-bertrand_gradualizing_2022}
  \item A reasonably gradual type theory, ICFP
    2022.\cite{maillard_reasonably_2022}
  \item Propositional equality for gradual dependently typed programming, ICFP
    2022.\cite{eremondi_propositional_2022}
  \item Partial Gradual Dependent Type Theory, SPLASH
    2023.\cite{shi_partial_2023}
  \item Gradual Indexed Inductive Types, ICFP 2024.\cite{malewski_gradual_2024}
\end{enumerate}

\subsection{Introduction}

Siek and Taha\cite{siek_gradual_2006} introduced the notion of gradual typing
which allows mixing of static and dynamic typing using a principled approach.
They introduced a new type \verb|?| which any term can inhabit similar to the
\verb|Any| type in dynamically-typed or uni-typed languages. Unlike
dynamically-typed languages, the programmer has control over which portions of
the program gets type-checked at run-time and which ones get checked at
compile-time by inserting \verb|?| into the type annotations. When the
type-checker checks the program, it elaborates the surface program written by a
programmer into a cast calculus by inserting casts. The casts ensure that the
run time performs necessary checks on values moving across the static-dynamic
boundary.

\citet{siek_refined_2015} noticed that the description of gradual types in
\citet{siek_gradual_2006} was not formalized rigorously which later resulted in
the creation of gradually typed systems which contain some undesirable
properties. Thus, they came up with the precision relation \(\sqsubseteq\) which
related types that are syntactically similar, except that some sub-terms in one
type correspond to the \verb|?| type in the other type. We say that a type
\verb|T|\(_1\) is more precise than another type \verb|T|\(_2\), i.e.
\verb|T|\(_1 \sqsubseteq\) \verb|T|\(_2\), when \verb|T|\(_2\) has more \verb|?|
in its syntax than \verb|T|\(_1\). For example, we can compare the precision of
two function types as \verb|T -> (T -> T)| \(\sqsubseteq\) \verb|T -> ?| where
\verb|T| could be any type. The notion of precision on types extends to terms as
well. We say that a term \verb|t|\(_1\) is more precise than a term
\verb|t|\(_2\), i.e. \verb|t|\(_1 \sqsubseteq\) \verb|t|\(_2\), if and only if
they differ only in type annotations and for each type annotation \verb|T|\(_1\)
in \verb|t|\(_1\), \verb|T|\(_1 \sqsubseteq\) \verb|T|\(_2\), where
\verb|T|\(_2\) is the corresponding type annotation in \verb|t|\(_2\) . Using
this precision relation, \citet{siek_refined_2015} define the static and dynamic
\textit{gradual guarantees}.

With the precision relation \(\sqsubseteq\), \citet{siek_refined_2015} stated
the \textit{gradual guarantees} which were desirable for gradually typed
languages to fulfill. The gradual guarantees use the judgment
\verb|e|\(:\)\verb|T| which states that a closed term \verb|e| has type \verb|T|
and the judgment \verb|e|\(\Downarrow\)\verb|v| expresses that a term \verb|e|
evaluates to a value \verb|v| and \verb|e|\(\Uparrow\) means that \verb|e|
diverges during evaluation.

\textbf{Gradual guarantees: } Suppose \verb|e|\(\sqsubseteq\)\verb|e'| and
\verb|e|\(:\)\verb|T|.
  \begin{itemize}
  \item \verb|e'|\(:\)\verb|T'| and
    \verb|T|\(\sqsubseteq\)\verb|T'| for some \verb|T'|.
  \item If \verb|e|\(\Downarrow\)\verb|v|, then \verb|e'|\(\Downarrow\)\verb|v'|
    and \verb|v|\(\sqsubseteq\)\verb|v'|. If \verb|e|\(\Uparrow\) then
    \verb|e'|\(\Uparrow\).
  \item If \verb|e'|\(\Downarrow\)\verb|v'|, then \verb|e|\(\Downarrow\)\verb|v|
    where \verb|v|\(\sqsubseteq\)\verb|v'|, or evaluating \verb|e| causes a
    dynamic type error. If \verb|e'|\(\Uparrow\) then \verb|e|\(\Uparrow\), or
    evaluating \verb|e| causes a dynamic type error.
  \end{itemize}
 The first guarantee is the \textit{static gradual guarantee} (SGG), while the
 other two guarantees are the \textit{dynamic gradual guarantees} (DGG). These
 gradual guarantees essentially state that typing and evaluation should be
 \textit{monotone with respect to the precision relation}. This means that
 losing precision should not lead to static or dynamic errors.

\citet{siek_gradual_2006} gradualize the simply typed lambda calculus (STLC) by
adding \verb|?| to the type syntax. Later works show how we can gradualize other
statically-typed systems such as ownership types\cite{sergey_gradual_2012},
refinement types\cite{lehmann_gradual_2017}, security
types\cite{fennell_gradual_2013}\cite{toro_type-driven_2018}\cite{chen_quest_2024}
and session types\cite{igarashi_gradual_2017}. Here, we will discuss the
gradualization of full-spectrum dependent types based on type theories like
Martin-L\"of’s dependent type theory\cite{martin-lof_intuitionistic_1984} and
calculus of constructions\cite{coquand_calculus_1988}.

By the propositions as types interpretation of dependent types, gradualization
allows us to partially state propositions by inserting \verb|?| in the type. The
gradual type system then generates casts which ensure that the propositions
corresponding to the statically typed parts of the program hold true during
runtime when values flow between the static-dynamic boundary. This should make
programming using dependent types more accessible since programmers can have a
running program even when the correctness of programs is partially specified and
verified.

This literature review will go over the systems described in the list of papers
mentioned in the question above. We will discuss the following gradual dependent
type systems:
\begin{enumerate}
  \item GDTL (\underline{G}radual \underline{D}ependently-\underline{T}yped
    \underline{L}anguage) \cite{eremondi_approximate_2019}\cite{eremondi_design_2023}
  \item GCIC (\underline{G}radual \underline{C}alculus of \underline{I}nductive
    \underline{C}onstruction) and its variants \GCICG{}, \GCICS{} (read
    ``GCIC-shift'') and \GCICN{}\cite{lennon-bertrand_gradualizing_2022}
  \item GRIP\cite{maillard_reasonably_2022}
  \item GEq (read
    ``geek'')\cite{eremondi_propositional_2022}\cite{eremondi_design_2023}
  \item \GGEq{}\cite{eremondi_design_2023} (assumed to be read
    ``guarded geek'')
  \item PUNK\cite{malewski_gradual_2024}
  \item PGTT (\underline{P}artial \underline{G}radual Dependent \underline{T}ype
    \underline{T}heory)\cite{shi_partial_2023}
\end{enumerate}

In this document we will constantly be referring to a static dependently-typed
language which gets gradualized by adding \verb|?| to the static language to get
a gradual surface language. We will also be referring to an intermediate
language like the cast calculus which the gradual surface language elaborates
into. Following the typographic conventions in \citet{eremondi_design_2023}, we
will typeset static dependently-typed terms in \Scode{red san-serif font},
gradual surface terms in \Gcode{green italic serif font} and elaborated
intermediate terms in \Ccode{blue bold serif font}.

In 2019, \citet{eremondi_approximate_2019} introduced GDTL which was the first
system which gradualizes full-spectrum dependent types by adding the \Gcode{?}
type in its type system. GDTL was later refined in Eremondi's 2023 PhD. thesis
\cite{eremondi_design_2023}. GDTL follows the AGT (Abstracting Gradual Typing)
approach pioneered in \citet{garcia_abstracting_2016}. Later in 2022,
\citet{lennon-bertrand_gradualizing_2022} introduced GCIC which highlights the
tension between three properties expected from a gradual dependent type theory
and how to design type theory variants addressing this tension. Later in the
same year, \citet{maillard_reasonably_2022} extended one of the variants of GCIC
to develop GRIP which was better suited to be a proof assistant. Simultaneously,
\citet{eremondi_propositional_2022} presented GEq which addresses how to
gradualize the propositional equality type in a different variant of GCIC which
was better suited to be a programming language. Independent of the GCIC and GDTL
lines of research, \citet{shi_partial_2023} (2023) introduces PGTT, which frames
gradualization as a transformation between static non-dependently typed programs
and static dependently typed programs, rather than between dynamically typed
programs and static dependently typed programs. Finally in 2024,
\citet{malewski_gradual_2024} extend GRIP to incorporate indexed inductive types
and call this system PUNK.

%Dependently-typed systems evaluate terms during type-checking. The evaluation is
%thus expected to terminate without producing dynamic type errors in order to let
%the type-checking procedure terminate. Adding \Gcode{?}, however, introduces the
%side effects of divergence and cast errors during the type-checking phase which
%is in tension with dependent type-checking. GDTL solves this issue by using an
%evaluation procedure called approximate normalization which we will discuss
%later.
%
%While GDTL doesn't account for inductive types, GCIC addresses this by
%gradualizing the calculus of inductive
%constructions(CIC)\cite{coquand_calculus_1988}. GCIC addresses the divergence of
%type-checking issue by presenting the Fire Triangle of Graduality theorem. It
%states that in order to avoid divergence (i.e. achieve \textit{strong
%  normalization}), a gradual dependently typed system needs to give up on some
%other important properties, namely \textit{graduality} and
%\textit{conservativity}. New and Ahmed\cite{new_graduality_2018} introduce
%graduality as a semantic reformulation of the gradual guarantees. Meanwhile,
%conservativity of a gradual type system with respect to a static type system
%states that all terms in the gradual system in which \Gcode{?} doesn't occur
%should coincide with terms in the static system. Each variant of GCIC presented
%in \cite{lennon-bertrand_gradualizing_2022} gives up on one property mentioned
%in the Fire Triangle theorem. The variant which gives up strong normalization
%(\(\mathcal{N}\)) is \GCICG{}, the \GCICN{} variant gives up graduality
%(\(\mathcal{G}\)) and the \GCICS{} variant gives up conservativity with respect to
%CIC (\(\mathcal{C}\)\textsubscript{/CIC}). The cast calculus which GCIC
%elaborates to is called CastCIC. The cast calculi for \GCICG{}, \GCICN{} and \GCICS{}
%variants are \CCICG, \CCICN and \CCICS, respectively.
%
%Maillard et al.\cite{maillard_reasonably_2022} claim that \GCICN{} is the most
%appealing system based on GCIC in the context of proof assistants. This is
%because \GCICN{} ensures decidability of type-checking, (weak) canonicity, and
%ensures that it supports existing CIC developments and libraries due to it being
%conservative with respect to CIC. Even though \GCICN{} doesn't satisfy
%graduality globally, Maillard et al. propose a novel method to prove the
%graduality of a fragment of terms in \GCICN{} within \GCICN{} itself. They
%present a modified version of \GCICN{} called GRIP which adopts a two-layer
%structure: an impure hierarchy of types for gradual terms, and a pure sort of
%propositions that can refer to gradual terms and errors, but whose inhabitants
%cannot use errors or unknown terms. It is in this pure propositional sort that
%they define precision as a type similar to how propositional equality types
%internalize the equality relation in static dependent types.
%
%Even though GCIC provides gradual inductive types, it doesn't provide gradual
%indexed inductive types which are essential to define the propositional equality
%type. Eremondi et al. present GEq\cite{eremondi_propositional_2022} which adds
%gradual propositional equality to \GCICG{} while preserving its properties.
%Unlike GRIP which was designed to be used as a proof assistant, GEq's authors
%designed it to be used as programming language with dependent types. \GCICG{}
%was chosen as a base to develop GEq because the non-termination of type-checking
%in \GCICG{} is not as detrimental for programming languages as compared to proof
%assistants. Moreover, the graduality and conservativity with respect to CIC
%properties of \GCICG{} are important for programming languages. Graduality
%allows for ease of evolution of program from being dynamically typed to
%statically typed, and conservativity allows the programmer to use the full
%capabilities of CIC once all type annotations are static.
%
%Later in his thesis\cite{eremondi_design_2023}, Eremondi introduces \GGEq{}
%which modifies GEq in order to get back decidable type-checking. \GGEq{} uses
%approximate normalization from GDTL achieve decidable type-checking. The
%decidability result is proven by building a syntactic model of \GGEq{} in a
%variant of Guarded Type Theory (GTT). Eremondi conjectures that \GGEq{}
%preserves the gradual guarantees and conservativity properties of GEq. He claims
%that \GGEq{} avoids the Fire Triangle of Graduality theorem because it doesn't
%restrict terms in the ways that GCIC variants do in order to fulfill the gradual
%guarantees or conservatively extending CIC.
%
%GDTL, GCIC and GEq lack indexed inductive types, but they have a way to use a
%restricted version of it. PUNK, however, is the first system to fully support
%gradual indexed inductive types\cite{malewski_gradual_2024}. PUNK extends GRIP
%by adding indexed inductive types to it. While presenting PUNK, Malewski et al.
%highlight that there is no unique or generally better way to deal with casts on
%type indices at runtime: there are just different possible semantics for cast
%reduction, with different tradeoffs, just like there are different possible
%semantics for higher-order casts in standard gradually-typed
%languages\cite{siek_exploring_2009}. PUNK has 3 different semantics for reducing
%multiple consecutive casts, namely free, meet and forgetful. As multiple casts
%are applied during reduction to a constructor, PUNK accumulates the indices in
%the casts' types into an index accumulator. An index accumulator \Ccode{acc} is
%equipped with a term representing the empty accumulator \Ccode{\(\mathbbm{1}\)}
%and an operation \(i \otimes\)\Ccode{\(acc\)} to extend an accumulator
%\Ccode{\(acc\)} with an index \(i\). Malewski et al. parameterize PUNK
%semantics based on how index accumulator operations are defined in order to get
%the 3 different kinds of cast reduction semantics.
%
%PGTT adopts an approach which is different than the other systems mentioned
%here. Instead of allowing programs to range from dynamically typed to static
%dependently typed, PGTT allows programs to range from static non-dependently
%typed to static dependently typed. This means that PGTT restricts entirely
%unknown types and only permits dynamic terms on the type indices,. For example,
%the type \Gcode{Vec n ?} is valid in PGTT, but \Gcode{?} as a type is invalid.
%PGTT's restriction on entirely unknown types simplifies runtime type checks,
%allows embedding into a static, dependently typed language, and reduces the
%performance overhead induced by gradual typing.

Now that we've had a bird's eye view of gradual full-spectrum dependently-typed
systems in the current literature, the following sections delve deeper into
these systems. Each of the following sections talks about the key design choices
and their properties followed by the limitations and tradeoffs of the type
system.

\subsubsection{GDTL}

\citet{eremondi_approximate_2019} derive GDTL from a static dependently-typed
system called SDTL (\underline{S}tatic \underline{D}ependently \underline{T}yped
\underline{L}anguage) which is a bidirectional, call-by-value variant of the
predicative fragment of CC\(_\omega\)\cite{coquand_calculus_1988}. SDTL syntax
of terms (metavariable \Scode{t} represents terms and \Scode{T} represents terms
that are types) contains lambda abstraction (\Scode{\(\lambda\)x.t}),
application (\Scode{t\(_1\) t\(_2\)}), dependent function type (\Scode{(x :
  T\(_1\)) \(\to\) T\(_2\)}), type universes (\Scode{Type\(_\ell\)} is a
universe at level \(\ell\)) and type ascription (\Scode{t::T}). SDTL is
bidirectionally type-checked and the type-checking rules ensure that two types
are the same by comparing their \(\alpha\)-equivalent, \(\beta\)-reduced,
\(\eta\)-long normal forms. The authors of GDTL then apply the AGT (Abstracting
Gradual Typing) methodology\cite{garcia_abstracting_2016} to SDTL to
systematically add \Gcode{?} to the static language. AGT alone is not enough to
incorporate the side-effects of divergence and type-mismatch errors, so the
authors of GDTL decided to use a separate notion of normalization during
type-checking called approximate normalization which terminates without raising
errors, but gives imprecise results. Approximate normalization is a major
contribution of GDTL and we will expand on it soon.

\subsubsection{Design Decisions and Their Properties}\label{subsec:gdtl_design_properties}

\paragraph{Introducing the unknown term}
A natural consequence of adding the \Gcode{?} type to a dependently-typed system
and the fact that dependent types are dependent on terms is that such a system
will also require us to add \Gcode{?} as a term. An example of why one would
need the \Gcode{?} term can be seen in the type of length-indexed lists, i.e.,
vectors. In a statically typed system, type of vectors of length \Scode{n}
containing elements of type \Scode{A} is \Scode{Vec A n}. A gradual dependent
type system allows us to express a vector whose size is unknown with the type
\Gcode{Vec A ?}. The second argument to \Gcode{Vec} should be a term with
natural number type which implies that \Gcode{?} in \Gcode{Vec A ?} has the
natural number type. Thus, even though \Gcode{?} occurs in a type it is actually
a term. GDTL and the other systems discussed in this literature review all have
\Gcode{?} as a term in some form.

\paragraph{Approximate hereditary substitution ensures termination of typechecking}

GDTL, being the first system to introduce gradual typing into a
dependently-typed system, addresses the major conflicts between the two typing
disciplines: divergence and error propagation during type-checking. In static
dependently-typed systems, usually the type-checker compares types to see if
they are equal. For example, when we apply a dependent function to an argument,
the domain type of the function and the type of the argument should be equal.
Two types are equal when they reduce to the same normal form. By adding
\Gcode{?} as both a type and a term, we can embed diverging terms from the
Untyped Lambda Calculus which will lead to non-termination during type-checking.

\begin{itemize}
  \item Describe adding Vec and Nat to GDTL in order to provide better examples.
  \item Example of a non-terminating type-checking program (Vector whose size diverges)
  \item Describe how GDTL can have type errors during normalization, even though
    the term being normalized is well-typed.
  \item Example of a program which causes a dynamic type error while normalizing
    during type-checking.
  \item Mention false gradual guarantee claim in
    \cite{eremondi_approximate_2019} which is why we will refer to Joey's thesis
    \cite{eremondi_design_2023} for discussing design of GDTL.
  \item Describe approximate normalization by mentioning peculiarities of normal
    form syntax, where is it used in the type-checking rules.
  \item Describe how approximate normalization handles non-termination.
  \item Describe how thesis version of approximate normalization handles dynamic
    type errors.
  \item Discuss the type preservation and termination properties of approximate
    normalization.
  \item Describe the run-time semantics of GDTL
    \begin{itemize}
    \item syntax uses evidence instead of ascriptions
    \item separate term for run-time errors
    \item values syntax prevents multiple evidence terms to be stacked on an actual value
    \item describe initial evidence and meet operation
    \item Semantics designed based on satisfying a hypothetical progress
      \& preservation property for the run-time language.
    \item Summarize section 3.6.3 in the thesis
    \end{itemize}
  \item Give an example of term being evaluated by the run-time rules.
  \item Discuss properties of GDTL
    \begin{itemize}
      \item type safety
      \item conservative extension of SDTL
      \item embedding the untyped lambda calculus
      \item gradual guarantees
    \end{itemize}
\end{itemize}
\subsubsection{Tradeoffs and Drawbacks}
\begin{enumerate}
  \item Hereditary substitution doesn't scale well for gradual typing because of
    the explosion of rules which get added in order to preserve termination
    whenever we add a new feature to the system.
  \item AGT was used in hopes of deriving a systematic way to derive a gradual
    dependently-typed language. Though it succeeds in helping us derive
    metafunctions, it doesn't seem to give us a straightforward way to derive
    approximate normalization which is a key feature.
  \item Approximate normalization results in a \Gcode{?} either if the
    type-checker decides that a function application diverges or if there's a
    type mismatch. Thus, when the normalization process gives us a \Gcode{?}, it
    becomes difficult to identify why normalization failed.
  \item GDTL doesn't support features that a practical dependently typed
    programming language needs.
\end{enumerate}

\subsubsection{GCIC}
\subsubsection{Design Decisions and Their Properties}
\begin{itemize}
  \item Lennon-Bertrand, Maillard and
    Tabareau\cite{lennon-bertrand_gradualizing_2022} design GCIC by gradualizing
    CIC\cite{coquand_calculus_1988} and explore various metatheoretic challenges
    associated with this endeavor.
  \item GCIC uses ideas from exceptional type theory
    (ExTT)\cite{pedrot_failure_2018}, embedding-projection
    pairs\cite{new_graduality_2018} and
    GDTL\cite{eremondi_approximate_2019}\cite{eremondi_design_2023} to extend
    CIC with the \verb|?| type.
  \item Lennon-Bertrand et al.\cite{lennon-bertrand_gradualizing_2022} emphasize
    how the (strong) \textit{Normalization} (\(\mathcal{N}\)) (similar to
    termination in GDTL) and (type) \textit{Safety} (\(\mathcal{S}\))
    (antithetical to cast errors in GDTL) properties of CIC get endangered when
    gradualizing it.
  \item Other desirable properties which GCIC should have are conservativity w.r.t.
    CIC (\(\mathcal{C}\)\textsubscript{/CIC}) and gradual guarantees.
  \item Graduality
    \begin{itemize}
      \item Describe how GCIC defines DGG as satisfying observational error
        approximation (Definitions 2 \& 3
        \cite{lennon-bertrand_gradualizing_2022})
      \item GCIC aims to satisfy \textit{Graduality} (\(\mathcal{G}\)) which
        subsumes DGG.
      \item Lennon-Bertrand et al. use the term \textit{Graduality}
        (\(\mathcal{G}\)) for the DGG established with respect to a notion of
        precision that also induces \textit{embedding-projection pairs} (ep-pairs).
      \item New and Ahmed\cite{new_graduality_2018} introduce ep-pairs in order to
        provide a more sematic formulation and proof of graduality.
      \item Lennon-Bertrand et al. state that, according to Max and New, the casts
        between two types related by precision form an ep-pair if and only if the
        casts form an adjunction which induces a retraction. This means that
        casting a term into a less precise type and back should give back the same
        term. For example, \Gcode{1::?::\(\mathbb{N}\)} should be equivalent to
        \Gcode{1}. Here, \Gcode{::} is the type ascription operator which gets
        converted to casts in the cast calculus.
      \item Lennon-Bertrand et al. claim that this alternate version of DGG ensures
        that type-checking mechanisms in the dynamic semantics such as casts only
        perform checking of types and do not alter the run-time behavior.
    \end{itemize}
  \item BCIC
  \item fire triangle theorem
  \item Describe use cases of the three GCIC variants
  \item Describe how the three variants get parameterized so that the same set
    of inference rules get reused
  \item Syntax of surface language
  \item Syntax of cast language
  \item Reduction rules of CastCIC
    \begin{itemize}
      \item IND-GERM and PROD-GERM rules are defined according to the discrete model.
    \end{itemize}
  \item Elaboration and its properties
    \begin{itemize}
      \item Usually elaboration judgment uses types from the source language,
        but here types are used from the target language i.e. cast calculus
      \item The CastCIC types used in the elaboration however are in one-to-one
        correspondence with GCIC types.
      \item Therefore, the \(\alpha\)-consistency relation which is used in the
        elaboration judgment ignores casts so that \(\alpha\)-consistency on
        CastCIC terms corresponds to \(\alpha\)-equality on GCIC terms which
        don't have casts.
    \end{itemize}
  \item Demonstrate how the diverging term \Ccode{\(\Omega\)} reduces in all
    variants
  \item Precision is a simulation for reduction
  \item Properties of GCIC and models built to prove them
\end{itemize}

\subsubsection{Tradeoffs and Drawbacks}
\begin{itemize}
  \item Writing down universe levels is tedious. Lennon-Bertrand et al.
    recommend typical ambiguity \cite{harper_type_1991} as the solution.
    \item \(\eta\)-expansion is not addressed here (and in future papers).
\end{itemize}
\subsection{GRIP}
\subsubsection{Design Decisions and Their Properties}
\subsubsection{Tradeoffs and Drawbacks}
\subsection{GEq}
\subsubsection{Design Decisions and Their Properties}
\citet{eremondi_propositional_2022} extend \GCICG{} with the propositional
equality type \Gcode{=}, its constructor \Gcode{refl} and eliminator \Gcode{J}
to create GEq. GEq elaborates into CastEq where the constructor \Ccode{refl}
carries a witness term along with it which is at least as precise as the terms
being equated in its type. As the program evaluates, these witnesses track
partial type information from casts of the equality type and raises a runtime
error when the information indicates that inconsistent terms are being equated.
CastEq also has a composition operator \Ccode{\&} which takes two witnesses and
returns a witness term whhich is at least as precise as its inputs. The CastEq
reduction rules use \Ccode{\&} to update a witness term by composing it with
terms occurring in the equality type of a cast.

%TODO: shorten this para
To illustrate how the witness and \Ccode{\&} catch errors,
\citet{eremondi_propositional_2022} started by considering the fixed-length list
type \Gcode{FList A n} representing a list containing \(n\) elements of type
\Gcode{A} and it is defined as the dependent product \(\Gcmath{\Sigma\ x
  :\textit{List} A.\ \textit{length}\ x =_\mathbb{N}\ n}\). Now consider the
following two functions operating on \Gcode{FList}s:
\[\Gcmath{\textit{zip}:\Pi\ (n:\mathbb{N}).\
    \textit{FList}\ \textit{Float}\ n \to \textit{FList}\ \textit{Float}\ n
    \to \textit{FList}\ (\textit{Float}\times\textit{Float})\ n}\]
\[\Gcmath{\textit{take}:\Pi\ (n:\mathbb{N}) (m:\mathbb{N}).\
    \textit{FList}\ \textit{Float}\ (n+m) \to
    \textit{FList}\ \textit{Float}\ n}\] \Gcode{zip} takes two lists of the same
length containing floating point numbers and returns another list of the same
length containing element-wise pairs of the two input lists. \Gcode{take}
returns an \Gcode{FList} containing the first \(n\) elements of the input
\Gcode{FList} of length \(n + m\). Here are some imprecise lists which we will
use as arguments for our functions:
\[\Gcmath{\textit{list}_1\ :=\ ([1.1,2.2], \textit{refl}_?):\textit{FList Float } ?}
\quad \quad \quad \quad \Gcmath{\textit{list}_2\ :=\ (\textit{Cons}(1.1, ?),
  \textit{refl}_?):\textit{FList Float } ?}\] The second projection of
\(\Gcmath{\textit{list}_1}\) satisfies the type \(\Gcmath{2\ =_\mathbb{N}\ ?}\)
because the length of the first projection is 2 and \Gcode{2} is consistent with
\Gcode{?}. The first projection is an imprecise list of length \(\Gcmath{1+?}\),
so the second projection satisfies the type \(\Gcmath{1+?\ =_\mathbb{N}\ ?}\)
again because \Gcode{1+?} and \Gcode{?} are consistent. Thus the term
\(\Gcmath{\textit{zip}\ ?\ list_1\ list_2}\) also synthesizes the type
\Gcode{FList Float ?} because the length of its inputs and its first argument is
\Gcode{?}. Finally the term
\(\Gcmath{\textit{take}\ 3\ ?\ (\textit{zip}\ ?\ list_1\ list_2)}\) has the type
\Gcode{FList Float 3} because the type of
\(\Gcmath{\textit{zip}\ ?\ list_1\ list_2}\) is consistent with the type
\Gcode{FList Float (3+?)} which is the type of the third formal parameter of
\Gcode{take}. Without witness checks, the result of evaluating \Gcode{take}
would be the list \Gcode{Cons((1.1,1.1), ?)}. However, GEq instead raises an
error because the witnesses stored in elaborated \Ccode{refl} terms of
\Ccode{list}\(\Ccmath{_1}\) and \Ccode{list}\(\Ccmath{_2}\) get composed as they
get evaluated until they can no longer be composed with the types in the cast
introduced by the expected type of \Ccode{take 3}. Initially the witness of
\Ccode{refl} is \(\Ccmath{?_{\square}}\), but when it is used to construct
\Ccode{list}\(\Ccmath{_1}\) it needs to be cast into the type
\(\Ccmath{2\ =_\mathbb{N}\ ?_\square}\). The equality type cast then causes the
witness term \(\Ccmath{?_{\square}}\) to be composed with \Ccode{2} and
\(\Ccmath{?_\square}\) which are the terms in the cast. The result of
composition is \Ccode{2} because it is the most precise term and then \Ccode{2}
becomes the new witness of the \Ccode{refl} in \Ccode{list}\(\Ccmath{_1}\). The
witness of the second projection of the result of \Ccode{zip} is also \Ccode{2}
because it has been the most precise term while the second projection of
\Ccode{list}\(\Ccmath{_1}\) was going through various casts to compute the
result of \Ccode{zip}. Eventually the cast introduced by \Ccode{take 3} tries to
compose the witness \Ccode{2} with the term \Ccode{3+?} which comes from the
expected size of \Ccode{take}'s input list. This composition fails because
\Ccode{2} ,i.e. \Ccode{S(S(0))} and \Ccode{3+?}, i.e. \Ccode{S(S(S(?)))} are not
consistent.

\subsubsection{Tradeoffs and Drawbacks}
\subsection{\GGEq{}}
\subsubsection{Design Decisions and Their Properties}
\subsubsection{Tradeoffs and Drawbacks}
\subsection{PUNK}
\subsubsection{Design Decisions and Their Properties}
\subsubsection{Tradeoffs and Drawbacks}
\subsection{PGTT}
\subsubsection{Design Decisions and Their Properties}
PGTT follows a 2-step approach where a gradual dependently typed language
\(\lambda^S\) gets elaborated into an intermediate cast calculus \(\lambda^I\)
and then into a purely static dependently typed language \(\lambda^C\). Instead
of allowing \(?\) as a type or a term, \(\lambda^S\) only permits \(?\) as a
type index like \(\text{Vec}(\mathbb{N}, ?)\). However, \(\text{Vec}(?, 0)\) is
an invalid type because a type universe cannot contain a dynamic term. This
ensures that all values have a constructor head even if their type contains an
unknown index. \citet{shi_partial_2023} claims that allowing dynamic terms only
on the type indices ensures normalization without proof. Since the target
language is purely static, PGTT has simpler and more efficient runtime checks.

Unlike \citet{lennon-bertrand_gradualizing_2022} who interpret the \(?\)
term as a dependent sum to build a discrete model of GCIC for a proof, PGTT's
semantics directly elaborates types with unknown indices in \(\lambda^S\) like
\(\text{Vec}(\mathbb{N}, \text{suc}(?))\) into dependent sums like \(\Sigma
(i:\mathbb{N}).\text{Vec}(\mathbb{N}, \text{suc}(i))\) in \(\lambda^I\). To be
precise, they elaborate into telescopes so that a type with multiple unknown
indices like \(\text{Vec}(\text{Vec}(\mathbb{N}, \text{suc}(?)), ?)\) elaborates
into the telescope
\((x:\mathbb{N})\triangleright(y:\mathbb{N})\triangleright\text{Vec}(\text{Vec}(\mathbb{N},
\text{suc}(x)), y)\). During elaboration into \(\lambda^I\), if the type indices
are consistent but not equal, then the elaborator inserts a cast on the
telescope term. The elaborator from \(\lambda^I\) to \(\lambda^C\) transforms
casts into runtime checks producing an error monad value while also inserting
suitable monad operations like bind, return and fmap.

\subsubsection{Tradeoffs and Drawbacks}
PGTT gives up the ability to express fully dynamic programs in exchange for
useful properties like normalization and a better runtime.
\citet{shi_partial_2023} explains the elaboration process using an example with
a projection cast out of the dynamic term, so it is unclear how an injection
cast into \(?\) will be elaborated. A more detailed critique of PGTT seems
difficult because of the brevity of its description in \citet{shi_partial_2023}
which lacks any syntax, typechecking or elaboration rules.

\section{Implementing GCIC}\label{sec:question2}

\subsection{The Question}
Implement a type checker that matches as closely as you can one of the language
designs in the literature from Question 1. Test the type checker on all the
examples from the above papers that can be reasonably adapted. In what ways does
the type checker differ from the type system of the chosen language design? Is
there anything that makes the combination of gradual typing and dependent types
difficult to handle in the implementation of a type checker? Are there any
results in the literature that can help overcome those difficulties?

\subsection{About The GCIC Implementation}

Since most of the gradual dependent type theories in the current literature are
extensions of GCIC, I have decided to implement this system. Ideally, in the
future, I should be able to extend this implementation to implement GRIP, GEq,
\GGEq{} and PUNK. I have implemented GCIC using Racket. Even though my
implementation of GCIC accepts an s-expression based syntax, while referring to
the syntax, I will translate the s-expression syntax to a GCIC-like syntax
presented in \citet{lennon-bertrand_gradualizing_2022} for the rest of this
document. The source code for this implementation is at:
\url{https://github.com/DarshalShetty/QualExam/tree/main/GCIC-implementation}.
My implementation consists of three phases that an input GCIC program goes
through: parsing, elaboration and evaluation.

The parsing phase accepts a GCIC program and outputs an Abstract Syntax Tree
(AST). The \texttt{parse} function in \texttt{syntax.rkt} initiates this phase.
\texttt{parse} emits an AST only after checking that the input program is
syntactically valid.
%validates the syntax by checking that the
%main term in the input program has no free variables and contains syntactically
%valid inductive type instantiation, inductive term constructors and inductive
%eliminators. A syntactically valid inductive type instantiation is one which
%refers to an inductive type name declared in the sequence of inductive type
%definitions at the beginning of the program. A syntactically valid inductive
%term constructor is one which refers to a constructor name declared in one of
%the inductive types in the sequence of inductive type definitions at the
%beginning of the program. A syntactically valid inductive eliminator is one
%whose branches handle all the constructors of the inductive type whose term is being
%eliminated.
Here is an example of a GCIC program in the implementation which highlights how
this syntax differs from that of \citet{lennon-bertrand_gradualizing_2022}:
\begin{example}[A simple \GCICN{} program]
\label{ex:simple}\
\begin{lstlisting}[mathescape,language=GCIC]
variant gcic-n
data Nat := zero | succ (_ : Nat$_{@\{0\}}$()).
data Bool := true | false.

ind$_\texttt{Nat}$(succ$_{@\{0\}}$(zero$_{@\{0\}}$()), _. Bool$_{@\{0\}}$(), _,
  zero => true$_{@\{0\}}$(),
  succ(_) => false$_{@\{0\}}$())
\end{lstlisting}
\end{example}
\noindent The above \GCICN{} program defines the \Gcode{Nat} and \Gcode{Bool}
inductive types\footnote{These inductive definitions omit the return type of
constructors and the type of the inductive type itself because the lack of
indexed inductive types makes it obvious what the ommitted types should be.}.
%Since there are no indexed inductive types, we omit specifying the
%type of the inductive type and their constructors in the definition. Thus, the
%type of \Gcode{Nat} and \Gcode{Bool} are not mentioned in the inductive
%definition and we assume their type to be \(\Gcmath{\square}\). Similarly, we
%omit the type of the constructors \Gcode{zero} and \Gcode{succ} because they
%have to be of type \Gcode{Nat()} and type of constructors \Gcode{true} and
%\Gcode{false} have to be of type \Gcode{Bool()}. If the universe level is 0,
%then the implementation allows us to omit the \(\Gcmath{_{@\{0\}}}\) annotation
%from the inductive type and term constructors.
The term at the end of the program is eliminating the unary representation of
the natural number 1 by returning a boolean term for each of the \Gcode{Nat()}
constructors.

The elaboration phase performs bidirectional typechecking on the inductive
definitions and the main term and then returns a corresponding CastCIC term. The
\texttt{elab-program} function defined in \texttt{elaboration.rkt} initiates the
elaboration phase. \texttt{elab-program} follows the elaboration rules in Fig. 9
of \citet{lennon-bertrand_gradualizing_2022}. Running \texttt{elab-program} on
Example~\ref{ex:simple} produces the same program, so we will consider how
\texttt{elab-program} adds casts on the following slightly modified version of
Example~\ref{ex:simple}:
\begin{example}[Modified \GCICN{} program]
\label{ex:simple-unk}
The return type of the inductive eliminator is \Gcsub{?}{@\{1\}} and not
\Gcsub{Bool}{@\{0\}}\Gcode{()}.
\begin{lstlisting}[mathescape,language=GCIC]
variant gcic-n
data Nat := zero | succ (_ : Nat$_{@\{0\}}$()).
data Bool := true | false.

ind$_\texttt{Nat}$(succ$_{@\{0\}}$(zero$_{@\{0\}}$()), _. ?$_{@\{1\}}$, _,
  zero => true$_{@\{0\}}$(),
  succ(_) => false$_{@\{0\}}$())
\end{lstlisting}
\end{example}
\noindent Here is the result of invoking \texttt{elab-program} on
Example~\ref{ex:simple-unk}:
\begin{example}[Elaborated Example~\ref{ex:simple-unk}]
\label{ex:simple-unk-cast}\
\begin{lstlisting}[mathescape,language=CCIC]
variant gcic-n
data Nat := zero | succ (_ : Nat$_{@\{0\}}$()).
data Bool := true | false.

ind$_\texttt{Nat}$(succ$_{@\{0\}}$(zero$_{@\{0\}}$()), _. <$\square_0$ $\Leftarrow$  ?$_{\square_{1}}$> ?$_{\texttt{?}_{\square_{1}}}$, _,
  zero => <?$_{\square_{0}}$ $\Leftarrow$ Bool$_{@\{0\}}$()> true$_{@\{0\}}$(),
  succ(_) => <?$_{\square_{0}}$ $\Leftarrow$ Bool$_{@\{0\}}$()> false$_{@\{0\}}$())
\end{lstlisting}
\end{example}

\noindent Notice how casts gets introduced around \Ccode{true} and
\Ccode{false}. \texttt{elab-program} also inserts casts into inductive
definitions when the parameter or argument types contain the unknown term
\Gcode{?}. The implementation of elaboration rules also adds some optimizations
which we will discuss in Section \ref{sec:elab-optimize}.

The elaboration rules require normalized types for checking if two types are the
same. For example, consider the following function application term:
\[
\Gcmath{(\lambda x : ((\lambda t : \square_0.\, t)\ \text{\Gcode{Nat}}_{@\{0\}}()).\, x)\ \text{\Gcode{zero}}_{@\{0\}}()}
\]
For the above expression to typecheck and elaborate, the elaborated type of the
function parameter \Gcode{x} and the elaborated type of the function argument
\Gcsub{zero}{_{@\{0\}}}\Gcode{()} should be \(\alpha\)-equivalent. The
elaborated function parameter type \(\Gcmath{(\lambda t : \square_0.\, t)
  \text{\Gcode{Nat}}_{@\{0\}}()}\) and the function argument type
\Gcsub{Nat}{@\{0\}}\Gcode{()}, however, are not \(\alpha\)-equivalent unless we
normalize both types. The evaluation phase described next provides the
\texttt{normalize} function which performs both type normalization in this phase
and input program evaluation in the next phase.
%The CHECK elaboration rule requires us to check
%whether two types are consistent and we implement this check by ensuring that
%the normal forms of the CastCIC elaborations of the two types are
%\(\alpha\)-consistent. The implementation executes \(\alpha\)-consistency checks
%by following the rules in Fig. 8 of \cite{lennon-bertrand_gradualizing_2022}.
%The implementation computes normal forms of CastCIC types by following the
%normalization procedure performed in the normalization phase which we will
%discuss next.

The evaluation phase reduces the main CastCIC term generated by the elaboration
phase until there are no more redexes left in the reduced term which
\citet{gregoire_compiled_2002} refer to as \textit{strong reduction}. The
\texttt{normalize} function in \texttt{evaluation.rkt} initiates the evaluation
phase by performing strong reduction on the input CastCIC term. The
implementation of \texttt{normalize} is based on the \(\mathcal{N}\) function
from \citet{gregoire_compiled_2002}. \texttt{normalize} first performs
\textit{symbolic weak reduction} by calling the \texttt{evaluate} function
followed by a call to the \texttt{readback} function. Symbolic weak reduction
produces a head normal form where reductions are not performed under variable
binding sites like the body of a lambda or the body of an inductive elimination
case. While performing weak symbolic reduction, \texttt{evaluate} uses the
principle reductions from Fig. 5 of \citet{lennon-bertrand_gradualizing_2022}
along with principle reduction rules which handle neutral terms which are stuck
due to free variables. \texttt{evaluate} needs to deal with free variables
because \texttt{readback} invokes \texttt{normalize} under variable binders
which makes the bound variable occurrences seem like free variables. Instead of
naively running a decompose-step-plug loop, \texttt{evaluate} fuses the
plug-decompose portion of the loop into an efficient \texttt{refocus} function
as described by \citet{danvy_refocusing_2004}. The abstract machine described in
Section~4.4.2 of \citet{danvy_refocusing_2004} is the source of influence for
\texttt{evaluate} which is essentially a call-by-name version of the CK abstract
machine\cite{felleisen_control_1987}.

%takes a term in head normal form and invokes
%\texttt{normalize} under the binder for head normal forms that bind variables.
%Symbolic weak reduction also handles reduction of free variables and the neutral
%forms or stuck terms generated by these free variables. Symbolic weak reduction
%is implemented by using Danvy and Nielsen's refocusing
%procedure\cite{danvy_refocusing_2004} which makes the symbolic weak reduction
%implementation tail-recursive. The principal reduction rules follow from Fig. 5
%of \cite{lennon-bertrand_gradualizing_2022} along with more rules to handle free
%variables and construct a spine representing neutral terms stuck on a free
%variable. The readback process takes the result of symbolic weak reduction and
%converts spines into CastCIC terms with free variables by mutually recursively
%calling ize on sub-terms of a spine.
To better understand how \texttt{normalize} works, consider the following
function application term:
\[
\begin{aligned}
  &\Ccmath{(\lambda x : \text{\Ccode{Nat}}_{@\{0\}}() \to \text{\Ccode{Nat}}_{@\{0\}}() .\, x)}\\
  &\Ccmath{(\lambda y : \text{\Ccode{Nat}}_{@\{0\}}() .\,
    \text{\CastTwo{\Ccode{Nat}\(_{@\{0\}}\)\Ccode{()}}
      {\(?_{\square_0}\)}
      {\Ccode{Nat}\(_{@\{0\}}\)\Ccode{()}}
      {\Ccode{zero}\(_{@\{0\}}\)\Ccode{()}}})}
\end{aligned}
\]
Here we are essentially applying the identity function to the constant zero
function. The constant zero function not only returns \Ccode{zero} but also
attaches two consecutive casts to it: from \Ccode{Nat} to \Ccode{?} and from
\Ccode{?} back to \Ccode{Nat}. \texttt{normalize} reduces the above term to its
strong normal form by \(\beta\)-reducing the identity function and then removing
all casts by consecutively applying the UP-DOWN nd IND-IND rules from Fig.~5 of
\citet{lennon-bertrand_gradualizing_2022}. First, \texttt{normalize} invokes
\texttt{evaluate} with the term above produces the following term:
\[
\begin{aligned}
  &\Ccmath{(\lambda y : \text{\Ccode{Nat}}_{@\{0\}}() .\,
    \text{\CastTwo{\Ccode{Nat}\(_{@\{0\}}\)\Ccode{()}}
      {\(?_{\square_0}\)}
      {\Ccode{Nat}\(_{@\{0\}}\)\Ccode{()}}
      {\Ccode{zero}\(_{@\{0\}}\)\Ccode{()}}})}
\end{aligned}
\]
\texttt{evaluate} only applies the \(\beta\)-reduction. Since, symbolic weak
reduction doesn't apply reduction rules under binding sites and
\texttt{evaluate} implements symbolic weak reduction, it doesn't apply UP-DOWN
and IND-IND on the body of \(\Ccmath{\lambda y}\). Finally, \texttt{normalize}
invokes \texttt{readback} on the above head normal term produced by
\texttt{evaluate} to produce the following strong normal term:
\[
\begin{aligned}
  &\Ccmath{(\lambda y : \text{\Ccode{Nat}}_{@\{0\}}() .\,
    \text{\Ccode{zero}\(_{@\{0\}}\)\Ccode{()}})}
\end{aligned}
\]
\texttt{readback} recursively calls \texttt{normalize} on the body of
\(\Ccmath{\lambda y}\) which then applies the reduction rules to remove the
casts. If the variable \Gcode{y} occurred in the body of \(\Ccmath{\lambda y}\),
then the recursive call to \texttt{normalize} would treat it as a stuck term.

\subsection{Difference Between The Implementation And Specification Of GCIC}
\label{sec:impl-differences}

\subsubsection{Lack of well-formedness and termination checks}
The specification of GCIC by \citet{lennon-bertrand_gradualizing_2022} assumes
that the inductive types in GCIC are well-formed, but this GCIC implementation
doesn't check whether the inductive type definitions in the surface program are
well-formed. The means that we can define a non-well-formed inductive type in
this implementation which will lead to a logical inconsistency. Similarly, the
GCIC specification expects the recursive calls made by the fixpoint function
provided by an inductive eliminator to be guarded in order to prevent
non-termination. However, this GCIC implementation allows unguarded recursive
calls. This means that if a non-decreasing argument is provided to the recursive
call, the GCIC term will diverge even in the terminating variants \GCICN{} and
\GCICS{}. We can implement these checks in the future by following the work of
\citet{gimenez_structural_1998} as suggested by
\citet{lennon-bertrand_gradualizing_2022}.

\subsubsection{Absence of universe level polymorphism}
The elaboration rules IND and CONS specified in Fig. 9 of
\citet{lennon-bertrand_gradualizing_2022} use the metafunctions Params and Args
in their premise. These metafunctions accept a universe level as an argument.
Ideally, this universe level should instantiate a \textit{universe level
  variable} used in the inductive definition of the type associated with the
terms in the conclusions of the IND and CONS rules. Such universe level
polymorphism enables the reuse of the same inductive definition for terms which
refer to different universe levels. This GCIC implementation, however, doesn't
support universe level polymorphism.

For example, here is the inductive definition of the list type according to the
GCIC specification:
\begin{lstlisting}[mathescape,label=lst:list-def-i, language=GCIC]
data List (A: $\square_i$) := null $|$ cons (_: A) (_: List$_{@\{i\}}$(A))
\end{lstlisting}
The variable \Gcode{i} is the \textit{universe level variable} mentioned earlier
which GCIC can instantiate with a universe level from the set \(\mathbb{N}\).
While elaborating the singleton list containing zero
\Gcsub{cons}{@\{0\}}\Gcsub{(Nat}{@\{0\}}\Gcode{(),}
\Gcsub{zero}{@\{0\}}\Gcsub{(),null}{@\{0\}}\Gcsub{(Nat}{@\{0\}}\Gcode{()))}, the
Params and Args metafunctions will instantiate \Gcode{i} with \Gcode{0} because
of the subscript \Gcode{0} in \Gcsub{cons}{@\{0\}}. Similarly, \Gcode{i} is
instantiated with \Gcode{1} while elaborating the singleton list containing the
natural number type \Gcsub{cons}{@\{1\}}\(\Gcmath{(\square_{0}}\)\Gcode{,}
\Gcsub{Nat}{@\{0\}}\Gcsub{(),null}{@\{1\}}\(\Gcmath{(\square_{0}}\)\Gcode{))}.
Right now, the implementation only allows elements of \(\mathbb{N}\) to be used
as universe levels and doesn't allow polymorphic universe level variables like
\Gcode{i} in inductive definitions.

%This universe level is a
%natural number from the set \(\mathbb{N}\) and not a unary natural number of
%type \Gcode{Nat()}. GCIC instantiates \Gcode{i} when elaborating a list type
%like \Gcsub{List}{@\{0\}}\Gcsub{(Nat}{@\{0\}}\Gcode{())} (list of natural
%numbers) or a term of this type like
%\Gcsub{cons}{@\{0\}}\Gcsub{(Nat}{@\{0\}}\Gcode{(),}
%\Gcsub{zero}{@\{0\}}\Gcsub{(),null}{@\{0\}}\Gcsub{(Nat}{@\{0\}}\Gcode{()))}
%(singleton list containing the number 0).
%
%When the elaboration algorithm elaborates the term
%\Gcsub{cons}{@\{0\}}\Gcsub{(Nat}{@\{0\}}\Gcode{(),}
%\Gcsub{zero}{@\{0\}}\Gcsub{(),null}{@\{0\}}\Gcsub{(Nat}{@\{0\}}\Gcode{()))},
%first, the Params metafunction instantiates \Gcode{i} occurring in the parameter
%types(s) of \Gcode{List}'s inductive definition with 0 and checks whether the
%parameter input(s) passed to \Gcode{cons} matches the instantiated parameter
%type(s). Params instantiates \Gcode{i} to 0 because of the subscript \Gcode{0}
%in \Gcsub{cons}{@\{0\}}. Since the \Gcode{List} type definition has only one
%parameter \Gcode{A} with the type \Gcmath{\(\square_i\)}, Params checks whether
%the first input to \Gcode{cons}: \Gcsub{Nat}{@\{0\}}\Gcode{()}, has the
%instantiated type \Gcmath{\(\square_0\)}.
%
%Second, the elaboration algorithm invokes the metafunction Args to ensure that
%the next two inputs of \Gcode{cons}, i.e. its arguments, have the right type.
%This is done by again instantiating \Gcode{i} with 0 but this time in the
%argument types of the \Gcode{cons} constructor in the inductive definition of
%\Gcode{List}. Then Args ensures that the two arguments of \Gcode{cons}:
%\Gcsub{zero}{@\{0\}}\Gcode{()} and
%\Gcsub{null}{@\{0\}}\Gcsub{(Nat}{@\{0\}}\Gcode{())}, inhabit the argument types
%of \Gcode{cons} in the inductive definition after the necessary substitutions:
%\Gcsub{Nat}{@\{0\}}\Gcode{()} and
%\Gcsub{List}{@\{0\}}\Gcsub{(Nat}{@\{0\}}\Gcode{())}, respectively.
%
%Thus the constructor term \Gcsub{cons}{@\{0\}}\Gcsub{(Nat}{@\{0\}}\Gcode{(),}
%\Gcsub{zero}{@\{0\}}\Gcsub{(),null}{@\{0\}}\Gcsub{(Nat}{@\{0\}}\Gcode{()))}
%elaborates successfully because the type inhabitation checks made by the Params
%and Args metafunctions on the inputs of the constructor succeed after
%instantiating the universe level in the \Gcode{List} definition to 0. Due to
%universe level polymorphism, the GCIC specification states that we can
%successfully elaborate the constructor term
%\Gcsub{cons}{@\{1\}}\(\Gcmath{(\square_{0}}\)\Gcode{,}
%\Gcsub{Nat}{@\{0\}}\Gcsub{(),null}{@\{1\}}\(\Gcmath{(\square_{0}}\)\Gcode{))}
%(singleton list containing the type \Gcsub{Nat}{@\{0\}}) which inhabits the same
%\Gcode{List} type. This is possible because in this case Params and Args
%instantiate \Gcode{i} in the \Gcode{List} inductive definition to 1.
%
%Since the GCIC implementation doesn't have universe level polymorphism, we
%cannot use the \Gcode{List} definition above which uses a polymorphic universe
%level \Gcode{i}. Instead we would need multiple \Gcode{List} definitions with
%the only difference between them being a different number for \Gcode{i}. To
%successfully elaborate \Gcsub{cons}{@\{0\}}\Gcsub{(Nat}{@\{0\}}\Gcode{(),}
%\Gcsub{zero}{@\{0\}}\Gcsub{(),null}{@\{0\}}\Gcsub{(Nat}{@\{0\}}\Gcode{()))} we
%would need the following definition of \Gcode{List}:
%\begin{lstlisting}[mathescape,label=lst:list-def-0, language=GCIC]
%data List (A : $\square_0$) :=
%$|$ null
%$|$ cons (_ : A) (_ : List$_{@\{0\}}$(A))
%\end{lstlisting}
%This definition doesn't elaborate the term
%\Gcsub{cons}{@\{1\}}\(\Gcmath{(\square_{0}}\)\Gcode{,}
%\Gcsub{Nat}{@\{0\}}\Gcsub{(),null}{@\{1\}}\(\Gcmath{(\square_{0}}\)\Gcode{))}
%because the type of the parameter \Gcode{A}, \(\Gcmath{\square_0}\), does not
%inhabit the parameter of the constructor \(\Gcmath{\square_0}\). Instead, for
%the new term to elaborate successfully, \Gcode{List} needs to be redefined with
%universe level 1 as follows
%\begin{lstlisting}[mathescape,label=lst:list-def-1, language=GCIC]
%data List (A : $\square_1$) :=
%$|$ null
%$|$ cons (_ : A) (_ : List$_{@\{1\}}$(A))
%\end{lstlisting}

Since most example programs which I ran in this implementation don't need
universe level polymorphism, I didn't consider adding this polymorphism feature
crucial. In cases where polymorphism is needed, duplication of the inductive
definition with a different universe level suffices. An alternative to
implementing universe level polymorphism would be to add universe level lifting
operators as described by \citet{maillard_reasonably_2022} for GRIP.

\subsubsection{Fixing IND-IND reduction rule}\label{sec:fix-ind-ind}

The IND-IND cast reduction rule from Fig. 5 of
\citet{lennon-bertrand_gradualizing_2022} is ambiguous since it seems like the
metavariable \(\Ccmath{\overline{a'}}\) appears out of nowhere. In order to fix
this ambiguity, I took inspiration from
\citet{lennon-bertrand_bidirectional_2022} which has cast reduction rules for
the constructors of a specific list type. Generalizing the NIL-NIL and CONS-CONS
reduction rules from Fig. 10.4b of \citet{lennon-bertrand_bidirectional_2022}
for any inductive type constructor, I came up with the following updated IND-IND
reduction rule for my implementation
\[
\begin{aligned}
  &\text{\Cast{\(\Ccmath{I(\overline{a_2})}\)}
              {\(\Ccmath{I(\overline{a_1})}\)}
              {\(\Ccmath{c(\overline{a}, b_1, ..., b_n)}\)}}
  \rightsquigarrow \Ccmath{c(\overline{a_2}, b'_1, ..., b'_n)}\\
  &\text{with }
    \Ccmath{b_k'} := \text{\Cast{\(\Ccmath{\text{Args}_k(I,i,c)[\overline{a_2},\overline{b}]}\)}
                                {\(\Ccmath{\text{Args}_k(I,i,c)[\overline{a_1},\overline{b}]}\)}
                                {\(\Ccmath{b_k}\)}}
\end{aligned}
\]
The version of GCIC that uses the NIL-NIL and CONS-CONS reduction rules from
\citet{lennon-bertrand_bidirectional_2022} satisfies the same theorems as the
version specified by \citet{lennon-bertrand_gradualizing_2022}. I therefore
conjecture that my GCIC implementation, which employs a generalized IND-IND
reduction rule subsuming NIL-NIL and CONS-CONS, should also satisfy the theorems
stated in \citet{lennon-bertrand_gradualizing_2022}.

\subsubsection{Elaboration optimizations}\label{sec:elab-optimize}

This GCIC implementation optimizes some of the elaboration rules so that the
elaborated CastCIC term has to undergo fewer reduction steps when the
normalization phase performs symbolic weak reduction on it. We will be referring
to the elaboration rules from Fig. 9 of
\citet{lennon-bertrand_gradualizing_2022}. The optimizations tweak the only rule
for the checking judgment
\ElabCheck{\(\Ccmath{\Gamma}\)}{\Gcode{t}}{\Ccode{T}}{\Ccode{t}} and some of the
rules for the constrained synthesis judgment
\ElabConstrSynth{\(\Ccmath{\Gamma}\)}{\Gcode{t}}{\Ccode{t}}{\bullet}{\Ccode{T}}.

Here is the CHECK elaboration rule for the checking judgment
\begin{prooftree*}
\hypo{\text{\ElabSynth{\(\Ccmath{\Gamma}\)}
    {\(\Gcmath{\tilde{t}}\)}{\Ccode{t}}{\Ccode{T}}}}
\hypo{\Ccode{T}~\text{\(\Ccmath{\sim}\)}~\Ccode{S}}
\infer2[CHECK]{\text{\ElabCheck{\(\Ccmath{\Gamma}\)}
    {\(\Gcmath{\tilde{t}}\)}{\Ccode{S}}{\Cast{\Ccode{S}}{\Ccode{T}}{\Ccode{t}}}}}
\end{prooftree*}
Most of the time \Ccode{T}\(\Ccmath{~=_\alpha~}\)\Ccode{S}, i.e. \Ccode{T} and
\Ccode{S} are \(\alpha\)-equivalent. In this case the elaborated cast in the
conclusion \Cast{\Ccode{S}}{\Ccode{T}}{\Ccode{t}} would eventually reduce to
\Ccode{t}. To avoid applying cast reduction rules later during the evaluation
phase, the implementation directly elaborates \(\Gcmath{\tilde{t}}\) to
\(\Ccmath{t}\) if \Ccode{T}\(\Ccmath{~=_\alpha~}\)\Ccode{S}. This optimization
is equivalent to applying the following CHECKEQ rule before applying the CHECK
rule in the specification:
\begin{prooftree*}
\hypo{\text{\ElabSynth{\(\Ccmath{\Gamma}\)}
    {\(\Gcmath{\tilde{t}}\)}{\Ccode{t}}{\Ccode{T}}}}
\hypo{\Ccode{T}~\text{\(\Ccmath{=_\alpha}\)}~\Ccode{S}}
\infer2[CHECKEQ]{\text{\ElabCheck{\(\Ccmath{\Gamma}\)}
    {\(\Gcmath{\tilde{t}}\)}{\Ccode{S}}{\Ccode{t}}}}
\end{prooftree*}

The INF-PROD?, INF-UNIV? and INF-IND? elaboration rules for the constrained
synthesis judgment have the following form
\begin{prooftree*}
\hypo{\text{\ElabSynth{\(\Ccmath{\Gamma}\)}
    {\(\Gcmath{\tilde{t}}\)}{\Ccode{t}}{\Ccode{T}}}}
\hypo{\Ccode{T}~\text{\(\Ccmath{\leadsto^*}\)}~\Ccode{?}_\Ccode{j}}
\infer2[INF-*?]{\text{\ElabConstrSynth{\(\Ccmath{\Gamma}\)}
    {\(\Gcmath{\tilde{t}}\)}
    {\Cast{\(\Ccmath{\text{germ}_i~}\)\(h\)}{\Ccode{T}}{\Ccode{t}}}
    {h}
    {\(\Ccmath{\text{germ}_i~}\)\(h\)}}}
\end{prooftree*}
where \(h \in \) Head from Fig. 4 of \citet{lennon-bertrand_gradualizing_2022},
i.e. \(h\) could be either \(\square_i\), \(\Pi\) or \(I\). The elaborated cast
in the conclusion uses the type \Ccode{T} which is the result of the synthesis
in the premise. This cast will reduce \Ccode{T} to \(\Ccode{?}_\Ccode{j}\) later
during the evaluation phase because the premise says so. To avoid this reduction
again later, this implementation replaces \Ccode{T} with \(\Ccode{?}_\Ccode{j}\)
in the elaborated cast. Thus the optimized version of the INF-PROD?, INF-UNIV?
and INF-IND? elaboration rules has the following form
\begin{prooftree*}
\hypo{\text{\ElabSynth{\(\Ccmath{\Gamma}\)}
    {\(\Gcmath{\tilde{t}}\)}{\Ccode{t}}{\Ccode{T}}}}
\hypo{\Ccode{T}~\text{\(\Ccmath{\leadsto^*}\)}~\Ccode{?}_\Ccode{j}}
\infer2[INF-*?]{\text{\ElabConstrSynth{\(\Ccmath{\Gamma}\)}
    {\(\Gcmath{\tilde{t}}\)}
    {\Cast{\(\Ccmath{\text{germ}_i~}\)\(h\)}{\(\Ccode{?}_\Ccode{j}\)}{\Ccode{t}}}
    {h}
    {\(\Ccmath{\text{germ}_i~}\)\(h\)}}}
\end{prooftree*}


\subsection{Conclusion}

The main conclusion of this implementation exercise is that it is possible to
implement a GCIC interpreter with minor changes to the rules in
\citet{lennon-bertrand_gradualizing_2022}. The challenges I faced stemmed not
from the inherent complexity of combining gradual typing with dependent types,
but from my limited understanding of the GCIC specification, which addresses
that complexity, and my lack of familiarity with implementation techniques for
dependent types. After Meven Lennon-Bertrand, the primary author of
\citet{lennon-bertrand_gradualizing_2022}, helped me understand the GCIC
specification, the implementation was straight-forward.

I conjecture that my GCIC implementation follows the specification. Even though
I don't have a proof to back my claim, I have test programs in
\texttt{example.rkt} to support it. \texttt{example.rkt} contains examples from
\citet{lennon-bertrand_gradualizing_2022} along with some examples I came up
with. The tests in the \texttt{tests} directory make sure that the examples are
consistent with how \citet{lennon-bertrand_gradualizing_2022} describe them to
behave.

%Since the authors of GCIC handle the complexity of the combination of gradual
%typing and dependent types in \cite{lennon-bertrand_gradualizing_2022} through
%the design decisions for GCIC, I didn't have to deal with handling this
%complexity in my implementation as long as I followed their specification. The
%major challenges I faced, however, were due to my lack of understanding of the
%subtleties in their specification of GCIC and my lack of knowledge about
%techniques to implement dependent types which the authors didn't cover since it
%was out of the scope of their paper. Fortunately, the authors of GCIC answered
%any questions I had about the subtleties of their specification over e-mail.
%Also, other researchers had already published literature which addressed
%solutions to any problems I had about implementation techniques which were not
%addressed in \cite{lennon-bertrand_gradualizing_2022}.
%
%The authors helped to clarify the role of the metafunctions Params and Args in
%enabling universe level polymorphism. They also assured me that the
%modifications described in Section~\ref{sec:fix-ind-ind} and
%Section~\ref{sec:elab-optimize} will still satisfy the necessary properties of
%GCIC. Danvy and Nielsen's work\cite{danvy_refocusing_2004} helped solve the
%problem of implementing congruence reduction rules in a reasonably efficient
%manner through refocusing. The GCIC specification mentions full reduction, but
%doesn't provide enough detail for me to implement it. However, Gr\'egoire and
%Leroy's work\cite{gregoire_compiled_2002} served as a foundation to implement
%full reduction for GCIC since they describe how to implement full reduction
%(strong reduction) for Calculus of Constructions.

\section{Analyzing The Practicality Of The Implementation}\label{sec:question3}

\subsection{The Question}
Critique the current state of the art with respect to the practical concern of
developing code and proofs about that code. Using the type checker that you
developed in question 2, apply it as best you can to developing the code and
correctness proof for a textbook data-structure of your choice (but more complex
than a linked-list or vector). Reflect on what went well versus what problems
were encountered.

\subsection{Introduction}

Since GCIC was one of the first gradual dependent type systems, it lacks many
features which are provided by current proof assistants. There are papers in the
current literature which address some of these features, but the other features
--- some of which mentioned as future work in the current literature --- have
not been explored yet. As a result of my GCIC implementation lacking features
which I discuss later, I was only able to implement vectors and nothing more
complicated. However, just by implementing vectors I was able to identify pain
points which would also trouble us when we implement advanced data structures
like binary search trees, red-black trees and tensors.

Before moving on to what went wrong with implementing vectors,
Section~\ref{sec:good-parts} discusses what went well. Every following section
discusses a different collection of problems.
Section~\ref{sec:literature-solutions} addresses two problems with this GCIC
implementation which some papers mentioned in Question~\ref{sec:question1} have
already explored. The remaining two sections mention problems which have either
been mentioned in the future work section of papers in
Question~\ref{sec:question1}, or have been solved for either dependently typed
or gradually typed programming languages, but not both.
Section~\ref{sec:ease-of-use} discusses problems which makes development of
programs and their correctness proofs in my GCIC implementation difficult.
Finally, Section~\ref{sec:efficiency} covers problems which I need to solve in
order to make runtime evaluation of GCIC programs in the implementation faster.

\subsection{The Good Parts}\label{sec:good-parts}
I was glad to realize that the GCIC implementation achieves gradual typing's
goal of allowing incremental development from untyped programs to fully typed
programs even in a dependently typed setting. The key advantage of this
incremental development style is that it allows me to suspend defining a
function on vectors (e.g., append) at any stage, deferring the proofs about
vector lengths by inserting the placeholder term \Gcode{?}. Such an incomplete
function would typecheck while still producing a vector value at runtime but
with some missing subterms represented by \Gcode{?}. This function can be used
to write test cases checking the computational result of the function while
ignoring the proof content which I can fill in later. Agda allows this kind of
incremental development using holes instead of the unknown term \Gcode{?}. Coq
doesn't have holes, but the closest it can get to this kind of incremental
development is by using the \texttt{Admitted} tactic for incomplete proofs.
Using \texttt{Admitted} is worse because it causes computation of the incomplete
function on vectors to be stuck on \texttt{Admitted} proof terms even when there
are no free variables.

Along with leaving the proofs terms incomplete in a function definition, GCIC
also allowed me to leave the specification of the function incomplete by using
\Gcode{?} in its type declaration. Consider the following incomplete type
declaration of the \Gcode{filter}\footnote{Defined as \texttt{fun-filter-f} in
\texttt{examples.rkt}} function from Example 1 of
\citet{lennon-bertrand_gradualizing_2022}\footnote{For brevity I will be using
\(\Gcmath{\mathbb{N}}\) and \(\Gcmath{\mathbb{B}}\) to refer to the types
\Gcode{Nat()} and \Gcode{Bool()}, respectively mentioned in
Question~\ref{sec:question2}. Also, I will be assuming that the implementation
can expand \(\lambda\) and \(\Pi\) terms with multiple parameters into their
curried forms.}:
\[
\Gcmath{\textit{filter} : \Pi\ (A: \square)\ (n: \mathbb{N})\ (f : A \to \mathbb{B}).\
  \textit{VecF}(A,\ n) \to \textit{VecF}(A,\ ?)}
\]
Since the length of the output vector is unspecified, it allows me to skip the
proofs about vector lengths as well in the definition of \Gcode{filter}. Agda
doesn't allow leaving the type declaration unspecified using a hole because the
typechecker complains about an unsolved constraint corresponding to the hole.
Another aspect in which holes in Agda are lacking is that the Agda compiler
fails to compile programs with holes to Haskell while programs with \Gcode{?}
have well-defined runtime semantics which should allow \Gcode{?} to be compiled
by a compiler in the future. I couldn't figure out a way by which one could have
an incomplete type declaration in Coq.

\subsection{Problems Which Have Solutions Available In The Literature}
\label{sec:literature-solutions}
\paragraph{Indexed Inductive Types}
This implementation allows us to define inductive types but not indexed
inductive types. Indexed inductive types allow us to specify more precise
constraints on our inductive types. For example this is how one would
define the vector type in a modern proof assistant:
\begin{lstlisting}[mathescape,language=GCIC]
data Vec (A: $\square$) : $\mathbb{N}$ $\to$ $\square$ :=
| nil: Vec(A, zero())
| cons (_: A) (n: $\mathbb{N}$) (_: Vec(A, n)): Vec(A, (succ(n))).
\end{lstlisting}
The current implementation can't express such an inductive definition because
the index \(\Gcmath{\mathbb{N}}\) in \(\Gcmath{\mathbb{N} \to \square}\) is not
supported by GCIC. This means that we can't use \(\Gcmath{\mathbb{N}}\)
constructors in the return type of our vector constructors. Instead I used
fording to define vectors in our implementation as follows:
\begin{lstlisting}[mathescape,language=GCIC]
data VecF (A: $\square$) (n: $\mathbb{N}$) : $\square$ :=
| nil-f (_: EqNat(zero(), n))
| cons-f (_: A) (m: $\mathbb{N}$) (_: EqNat(succ(m), n)) (_: VecF(A, m)).
\end{lstlisting}
Here, the index constraint on the vector constructor return types are replaced
by a constructor argument of type \Gcode{EqNat}\footnote{Defined as
\texttt{ind-def-eq-nat} in \texttt{examples.rkt}} representing a decidable proof
of equality between natural numbers. Notice how the index
\(\Gcmath{\mathbb{N}}\) is now the type parameter \(\Gcmath{n: \mathbb{N}}\).
Also this implementation allows us to skip the return type for the vector
constructors because without indexed inductive types the return type of both
vector constructors can only be \Gcode{VecF(A, n)}.

%Indexed inductive types are convenient because defining functions such as
%\Gcode{head} on vectors become easier. \Gcode{head} takes a \Gcode{Vec(A,
%  succ(n))} and returns the first element of the input vector. When defining
%\Gcode{head} in the GCIC implementation using the forded \Gcode{VecF}, I had to
%do an inductive elimination on the input vector. In the \Gcode{nil-f} case of
%the elimination, I had to prove a contradiction because the input vector's
%length is \Gcode{succ(n)} while the length of \Gcode{nil-f} is \Gcode{zero()}
%which can never be the same. If \Gcode{head} were to be defined using the
%indexed inductive type \Gcode{Vec}, the elimination form would not have to
%mention the \Gcode{nil} case because the return type of \Gcode{nil} would help
%the typechecker infer that this case is redundant. The removal of a single
%contradiction proof for \Gcode{head} due to an indexed inductive type might seem
%trivial, but using \Gcode{Vec} instead of \Gcode{VecF} in my implementation of
%vectors would help me skip a significant number of similar contradiction proofs
%in other functions on vectors and proofs about vectors.

\citet{malewski_gradual_2024} discuss how to add indexed inductive types to
\GCICN{} through PUNK. In the future, however, I would like to use GCIC as a
programming language and not a proof assistant so I would have to focus on
improving the \GCICG{} variant of the implementation which is suitable for this
purpose. Even if I add indexed inductive types to \GCICG{} by following PUNK, I
don't know whether it will lead to weird consequences since PUNK was supposed to
work with \GCICN{}.

\paragraph{Propositional Equality Type}
The propositional equality type (denoted by the binary relation \Gcode{=} is
what proof assistants use to specify equality constraints. Since GCIC doesn't
have a propositional equality type, the definition of \Gcode{VecF} uses the
decidable equality type on natural numbers \Gcode{EqNat}. It's called a
decidable equality because for any natural numbers \Gcode{m} and \Gcode{n},
\Gcode{EqNat(m, n)} evaluates to the type \Gcode{Unit} if \Gcode{m} and
\Gcode{n} contain the same number of \Gcode{succ} constructors, otherwise it
evaluates to the \Gcode{Void} type. Without propositional equality, I will have
to tediously define a different equality type for every inductive type. Adding
propositional equality would also allow me to define a substitution operation of
type \(\Gcmath{\Pi (A: \square) (x\ y: A) (P: A \to \square).\ x \equiv y \to
  P(x) \to P(y)}\). Because of decidable equality I had to define a different
substitution operation for different \Gcode{P}. For example, I had to define two
different substitution operations over \Gcode{EqNat}:
\[\Gcmath{\textit{subst-vec-f}: \Pi (x\ y: \mathbb{N}) (A: \square).\
  \textit{EqNat}(x,\ y) \to \textit{VecF}(A,\ x) \to \textit{VecF}(A,\ y)}\]
\[\Gcmath{\textit{subst-fin-f}: \Pi (x\ y: \mathbb{N}).\
  \textit{EqNat}(x,\ y) \to \textit{FinF}(x) \to \textit{FinF}(y)}\] Here,
\Gcode{FinF}\footnote{Defined as \texttt{ind-def-fin-f} in
\texttt{examples.rkt}} is the forded version of the finite set type \texttt{Fin}
in Agda or Coq. Notice how \Gcode{P} is instantiated with \(\Gcmath{\lambda n:
  \mathbb{N}.}\) \Gcode{VecF(A, n)} in \Gcode{subst-vec-f}\footnote{Defined as
\texttt{proof-subst-vec-f} in \texttt{examples.rkt}} and with \Gcode{FinF} in
\Gcode{subst-fin-f}\footnote{Defined as \texttt{proof-subst-fin-f} in
\texttt{examples.rkt}}. I believe that adding propositional equality types to
the \GCICG{} implementation as described in \citet{eremondi_propositional_2022}
should be straight forward since GEq extends \GCICG{}.

\subsection{Ease-of-use Problems}\label{sec:ease-of-use}
\paragraph{Lack Of Control Over Definition Unfolding}
I tried to emulate global definitions using an implementation-level let-binding
macro which expands to an application of a \(\lambda\)-abstraction. However,
this kind of a definition fails to typecheck in some cases. For example, I tried
to implement the type-level fixpoint version for vectors
\(\Gcmath{\textit{Vec}_\mu}\)\footnote{Defined as \texttt{fun-vec-}\(\mu\) in
\texttt{examples.rkt}} as described in section 7.1 of
\citet{lennon-bertrand_gradualizing_2022}. \(\Gcmath{\textit{Vec}_\mu}\) is a
function defined by recursion on its second argument such that the term
\(\Gcmath{\textit{Vec}_\mu\ A\ n}\) evaluates to an \(n\)-ary nested product
type. The problem with defining \(\Gcmath{\textit{Vec}_\mu}\) as a let-bound
identifier is that when I want to eliminate a term of type
\(\Gcmath{\textit{Vec}_\mu\ A\ n}\), I have to use the inductive eliminator for
the product type. The typechecker then fails to check the eliminator expression
because it expects the scrutinee type to be a product, but it is unable to
unfold the definition of \(\Gcmath{\textit{Vec}_\mu}\) in the synthesized type
of the scrutinee. Here, \(\Gcmath{\textit{Vec}_\mu\ A\ n}\) fails to normalize
to a product type because \(\Gcmath{\textit{Vec}_\mu}\) is a free variable
during typechecking. Thus, to satisfy the typechecker, I sparingly use the
let-binding macro. However, this makes debugging very difficult because then I
have to look at fully unfolded anonymous lambdas without knowing where they came
from or what they are computing.

\paragraph{Need For Blame Tracking}
Another feature that is essential for debugging is blame
tracking\cite{wadler_well-typed_2009}. In the GCIC implementation, any error
arising due to a type mismatch during cast reduction just results in an error
term without any indication as to which location the error originated from. Such
an error is not helpful for debuggingd. For example, consider the following
functions along with the incomplete \Gcode{filter} function discussed in
Section~\ref{sec:good-parts}:
\[ \Gcmath{\textit{compose} : \Pi\ (A\ B\ C: \square).\
  (B \to C) \to (A \to B) \to A \to C} \]
\[ \Gcmath{\textit{head} : \Pi\ (A: \square)\ (n: \mathbb{N}).\
  \textit{VecF}(A,\ \textit{succ}(n)) \to A} \] \Gcode{compose}\footnote{Defined
as \texttt{fun-compose} in \texttt{examples.rkt}} composes two functions and
\Gcode{head}\footnote{Defined as \texttt{fun-head-f} in \texttt{examples.rkt}}
returns the first element of its input vector. Notice how \Gcode{filter} is the
only function with a \Gcode{?} in its type. Now consider the following term
which uses \Gcode{filter}, \Gcode{compose} and \Gcode{head}:
\[ \Gcmath{\textit{compose}\ (\textit{VecF}(\mathbb{N},\ \underline{3}))\
                             (\textit{VecF}(\mathbb{N},\ ?))\ \mathbb{N}\
           (\textit{head}\ \mathbb{N}\ ?)\
           (\textit{filter}\ \mathbb{N}\ \underline{3}\
              (\lambda x:\mathbb{N}.\ x < \underline{1}))\
           \underline{[1;\ 2;\ 3]}} \]

Here, \(\Gcmath{\underline{n}}\) represents the number \(n\) constructed using
the \(\Gcmath{\mathbb{N}}\) constructors \Gcode{succ} and \Gcode{zero}.
Similarly, \(\Gcmath{\underline{[1;\ 2;\ 3]}}\) represents the corresponding
GCIC vector of type \(\Gcmath{\textit{VecF}(\mathbb{N},\ \underline{3})}\). The
above term applies a composed function returned by \Gcode{compose} to the vector
\(\Gcmath{\underline{[1;\ 2;\ 3]}}\). The composed function first applies
\Gcode{filter} and then \Gcode{head} to its argument. The result of evaluating
the above term after elaboration is the CastCIC error term
\(\Ccmath{\textit{err}_\mathbb{N}}\). The evaluation results in this error
because the result of \Gcode{filter} is the empty vector \Gcode{nil-f} of type
\(\Gcmath{\textit{VecF}(\mathbb{N},\ \textit{zero}())}\) while the input type of
\(\Gcmath{\textit{head}\ \mathbb{N}\ ?}\) is
\(\Gcmath{\textit{VecF}(\mathbb{N},\ \textit{succ}(?))}\) and obviously
\Gcode{zero()} and \Gcode{succ(?)} are not equal. An ideal blame tracking system
should blame the result of \Gcode{filter} when the evaluator passes it to
\Gcode{head} as an argument inside the definition of \Gcode{compose}. The blame
tracking system has to blame \Gcode{filter} because it is the only function
which has a \Gcode{?} in its type. \citet{wadler_well-typed_2009},
\citet{siek_blame_2021} and \citet{siek_refined_2015} show how to implement
blame tracking for a non-dependent gradually typed language and I will need to
figure out how to apply their blame tracking technique to my GCIC
implementation.

\paragraph{Need For Agda-like Dependent Pattern Matching}
My speed of development in the GCIC implementation was slow because I had to
manage the type dependency between the formal parameters of a dependently typed
function before eliminating them. This problem worsened as the number of
dependently typed formal parameters which I had to eliminate increased. For
example, here is the outline of the proof term for the \Gcode{subst-vec-f}
function mentioned above in Section~\ref{sec:literature-solutions}:
\begin{lstlisting}[mathescape,language=GCIC]
$\lambda$ m: $\mathbb{N}$.
 ind$_\mathbb{N}$(m, $...$, subst-rec,
      zero =>
        $\lambda$ n: $\mathbb{N}$.
         ind$_\mathbb{N}$(n, $...$, _,
              zero => $\lambda$ (A: $...$) (p: $...$) (v: $...$). $...$,
              succ(n$'$) => $\lambda$ (A: $...$) (p: $...$) (v: $...$). $...$),
      succ(m$'$) =>
        $\lambda$ n: $\mathbb{N}$.
         ind$_\mathbb{N}$(n, $...$, _,
              zero => $\lambda$ (A: $...$) (p: $...$) (v: $...$). $...$,
              succ(n$'$) => $\lambda$ (A: $...$) (p: $...$) (v: $...$). $...$))
\end{lstlisting}
The nesting of lambdas above is necessary because the type annotations of their
formal parameters (omitted here) need to refer to the elimination pattern
variables \(\Gcmath{m'}\) and \(\Gcmath{n'}\). With dependent pattern matching I
would be able to express the above proof term succinctly without any nesting.
For example, here is the comparatively shorter outline of \Gcode{subst-vec-f}
defined in Agda which has dependent pattern matching:
\begin{lstlisting}[mathescape]
SubstVecF zero zero A p v = $...$
SubstVecF zero (suc n) A p v = $...$
SubstVecF (suc m) zero A p v = $...$
SubstVecF (suc m) (suc n) A p v = $...$
\end{lstlisting}
Figuring out how to apply this kind of pattern matching to my GCIC
implementation would allow one to write much shorter and clear programs.

\paragraph{Need For Implicit Arguments}
Another way to make our proofs shorter would be to implement implicit arguments
which allows one to skip the arguments of functions, inductive types and their
constructors. The type system is then able to infer the skipped arguments
through unification. I should be able to add implicit arguments to GCIC by
figuring out how to integrate type inference for non-dependent gradual type
systems\cite{siek_gradual_2008}\cite{garcia_principal_2015} and implicit
argument implementation techniques used in Agda and Coq.

\paragraph{Need For Unifying \Gcode{?} And Typed Holes}
In Section~\ref{sec:good-parts} I mentioned how \Gcode{?} is comparable to holes
in Agda, and how it is even better in some regards. \Gcode{?} is lacking,
however, only because it doesn't keep track of the type of variables available
in its scope. If we somehow associate the types of the variables within scope
with the type of \Gcode{?} then we should be able to use \Gcode{?} as an aid
while developing software incrementally just like how we use holes in Agda. This
would synergize with gradual typing's goal of developing software by
incrementally adding types.

\subsection{Efficiency Problems}\label{sec:efficiency}

\paragraph{Lack of Space-efficiency}
Space-efficiency of casts is important for a gradual programming language
implementation since it ensures a reasonable upper bound on the space occupied
by casts during the evaluation of a term\cite{herman_space-efficient_2010}.
Since GCIC evaluates terms during typechecking, space-efficiency is desirable
for both run-time and compile-time. \citet{lennon-bertrand_gradualizing_2022}
have not commented on the space-efficiency of GCIC, but I conjecture that GCIC
is not space-efficient. To make my GCIC implementation space-efficient, I was
considering following the approach of \citet{siek_blame_2021} by redefining GCIC
semantics using a (non-space-efficient) coercion calculus and then transforming
it into a space-efficient coercion calculus.

\paragraph{Call-by-name Evaluation Is Slow}
Before working on my GCIC implementation, I had come to the wrong conclusion
that the implementation must use the call-by-name evaluation order because I
misunderstood the head neutral form rule for function application in Figure~7 of
\citet{lennon-bertrand_gradualizing_2022}. Call-by-name semantics is inefficient
because it evaluates a term multiple times rather than evaluating a term once
and then reusing the term's value multiple times. In order to make my GCIC
implementation follow the more efficient call-by-value semantics, I would have
to modify the functions in my implementation which handle refocusing. Still, I
doubt that the PROD-UNK and PROD-ERR rules from Fig.~5 of
\citet{lennon-bertrand_gradualizing_2022} will work well with call-by-value
semantics because they created these rules inspired by \textit{call-by-name
  exceptions} from \citet{pedrot_failure_2018}.

\paragraph{Presence Of Redundant Terms At Runtime}
The only use of the decidable equality type \Gcode{EqNat} is to satisfy the
typechecker regarding the length of the forded vector type \Gcode{VecF}.
However, at runtime, the functions operating on terms of type \Gcode{VecF} only
care about the elements of the vector while the \Gcode{VecF} constructor
arguments of type \Gcode{EqNat} are only used in places not relevant to the
runtime. For example, the definition of the vector append\footnote{Defined as
\texttt{fun-vec-f-append} in \texttt{examples.rkt}} function in my GCIC
implementation only uses the \Gcode{EqNat} arguments of the \Gcode{cons-f} and
\Gcode{nil-f} constructors to either prove a contradiction indicating that an
inductive eliminator branch is unreachable at runtime, or as an argument to the
\Gcode{subst-vec-f} function. The \Gcode{subst-vec-f} function returns a copy of
its input vector, but with a different type in order to satisfy the typechecker.
Thus, the vector append function doesn't care about \Gcode{EqNat} terms stored
in its vector arguments at runtime. Not only do such redundant terms waste space
in memory at runtime, but, in some cases, they also cause the runtime to waste a
significant amount of CPU cycles to construct them.
\citet{tejiscak_erasure_2020} explains this problem in detail while providing
solutions for them in a non-gradual setting.

%\section{Appendix: Normalization of CastCIC}\label{sec:castcic-normalize}
%
%This appendix explains how the normalization phase of my GCIC implementation
%works.
%
%\subsection{Symbolic weak reduction}
%
%Symbolic weak reduction reduces CastCIC terms with free variables to a canonical
%form \(\Ccmath{v}\) defined as follows:
%\[
%\begin{aligned}
% \text{Canon} \ni \Ccmath{v} ::=&\ \Ccmath{\lambda\, x : v.t}\
%    |\ \Ccmath{c(\overline{v}, \overline{v})}\
%    |\ \Ccmath{\Pi\, x : v.t}\
%    |\ \Ccmath{\square_i}\ |\ \Ccode{I}\Ccmath{(\overline{v})}\ |\ \Spine{k}\\
%    |&\ \Ccmath{?_{V}}\ |\ \Ccmath{\Ccode{err}_{V}}\
%    |\ \text{\Cast{\(\Ccmath{?_{\square_i}}\)}{\(\Ccmath{\text{germ}_i\ h}\)}{\(\Ccmath{t}\)}}\\
% \Ccmath{V} ::=&\ \Ccmath{\square_i}\ |\ \Ccode{I}\Ccmath{(\overline{v})}\
%    |\ \Ccmath{?_{\square_i}}\ |\ \Ccmath{\Ccode{err}_{\square_i}}\\
% \text{Spine} \ni \Ccmath{k} ::=&\ \Ccmath{x}\ |\ \Spine{k}\,\Ccmath{t}\ |\
%    |\ \Ccode{ind}_\Ccode{I}\Ccode{(}\Spine{k}\Ccmath{,\, x.t,\, x.\overline{\overline{y}.t}}\Ccode{)}\
%    |\ \Ccmath{?_{\Spine{k}}}\ |\ \Ccmath{\Ccode{err}_{\Spine{k}}}\\
%    |&\ \Ccmath{\lambda\, x : \Spine{k}.t}\ |\ \Ccmath{\Pi\, x : \Spine{k}.t}\
%    |\ \Ccmath{c(\overline{v}, \Spine{k}, \overline{t}, \overline{t})}\
%    |\ \Ccmath{c(\overline{v}, \overline{v}, \Spine{k}, \overline{t})}\\
%    |&\ \Ccode{I}\Ccmath{(\overline{v}, \Spine{k}, \overline{t})}\
%    |\ \text{\Cast{\(\Ccmath{t}\)}{\(\Spine{k}\)}{\(\Ccmath{t}\)}}\
%    |\ \text{\Cast{\(\Ccmath{t}\)}{\(\Ccmath{?_{\square_i}}\)}{\(\Spine{k}\)}}\
%    |\ \text{\Cast{\(\Spine{k}\)}{\(\Ccmath{\square_i}\)}{\(\Ccmath{t}\)}}\\
%    |&\ \text{\Cast{\(\Spine{k}\)}{\(\Ccmath{\Pi\, x : v.t}\)}{\(\Ccmath{t}\)}}\
%    |\ \text{\Cast{\(\Ccmath{\Pi\, x : v.t}\)}{\(\Ccmath{\Pi\, x : v.t}\)}{\(\Spine{k}\)}}\\
%    |&\ \text{\Cast{\(\Spine{k}\)}{\(\Ccode{I}\Ccmath{(\overline{v})}\)}{\(\Ccmath{t}\)}}\
%    |\ \text{\Cast{\(\Ccode{I}\Ccmath{(\overline{v})}\)}{\(\Ccode{I}\Ccmath{(\overline{v})}\)}{\(\Spine{k}\)}}\
%\end{aligned}
%\]
%
%
%
%The function \textit{evaluate} implements symbolic weak reduction using Danvy
%and Nielsen's \textit{refocus} function\cite{danvy_refocusing_2004}.
%\textit{evaluate} uses the principal reduction judgment ``\Ccode{t} \(\leadsto\)
%\Ccode{t}'' and the canonical form judgment ``canonical \Ccode{t}'' from figures
%5 of \cite{lennon-bertrand_gradualizing_2022}. The \textit{evaluate} function
%takes a CastCIC term as input and outputs either a weak-head value satisfying
%the canonical judgment or returns a pair of a stuck term along with its
%surrounding evaluation context. Let TERM be the type of all CastCIC terms, CANON
%be the type of all CastCIC terms \Ccode{v} such that canonical \Ccode{v} and
%EVALCTX be the type of evaluation contexts. The syntax of evaluation contexts is
%derived from the canonical and neutral form syntax in figure 7 of
%\cite{lennon-bertrand_gradualizing_2022}.
%
%\[
%\begin{aligned}
%\textit{evaluate} : \text{Term}_\text{CastCIC} &\to \text{Canon} + (\text{EvalCtx} \times \text{Term}_\text{CastCIC})\\
%\textit{evaluate}(\Ccode{t}) &= \textit{refocus}(\Ccode{t}, [])
%\end{aligned}
%\]
%
%The \textit{evaluate} function calls the \textit{refocus} function along with an
%empty evaluation context. The \textit{refocus} function accepts a term and an
%evaluation context as input and provides the same type of output as
%\textit{evaluate}. This implementation defines \textit{refocus} in a mutual
%recursive manner along with \(\Rfa{}\) and \textit{iterate}. The definitions of
%\textit{refocus}, \textit{iterate} and \(\Rfa{}\) are inspired from section
%4.4.2 of \cite{danvy_refocusing_2004}.
%
%\[
%\begin{aligned}
%  \textit{refocus} : \text{TERM} \times \text{REDCONTEXT} &\to \text{CANON} +
%    (\text{EVALCTX} \times \text{TERM})\\
%  \textit{refocus}(\Ccode{v}, \mathcal{E}) &= \Rfa{}(\mathcal{E}, \Ccode{v})
%    & \text{when canonical \Ccode{v}}\\
%  \textit{refocus}(\Ccode{x}, \mathcal{E}) &= \textit{iterate}(\mathcal{E}, \Ccode{x})
%    & \text{when variable \Ccode{x}}\\
%  \textit{refocus}(\Ccode{t}_\Ccode{1}\ \Ccode{t}_\Ccode{2}, \mathcal{E})
%    &= \textit{refocus}(\Ccode{t}_\Ccode{1}, \mathcal{E}[[]\ \Ccode{t}_\Ccode{2}])\\
%  \textit{refocus}(\Ccode{ind}_\Ccode{I}\Ccode{(t, z.P, f.}\overline{\overline{\Ccode{y}}\Ccode{.t}}\Ccode{)}
%        , \mathcal{E})
%    &= \textit{refocus}(\Ccode{t},
%        \mathcal{E}[\Ccode{ind}_\Ccode{I}\Ccode{(} []
%          \Ccode{, z.P, f.}\overline{\overline{\Ccode{y}}\Ccode{.t}}\Ccode{)}])\\
%  \textit{refocus}(\Ccode{?}_\Ccode{t}, \mathcal{E})
%    &= \textit{refocus}(\Ccode{t}, \mathcal{E}[\Ccode{?}_{[]}])\\
%  \textit{refocus}(\Ccode{err}_\Ccode{t}, \mathcal{E})
%    &= \textit{refocus}(\Ccode{t}, \mathcal{E}[\Ccode{err}_{[]}])\\
%  \textit{refocus}(\text{\Cast{\Ccode{t}}{\(\Ccode{t}_\Ccode{1}\)}{\(\Ccode{t}_\Ccode{2}\)}}, \mathcal{E})
%    &= \textit{refocus}(\Ccode{t}_\Ccode{1}, \mathcal{E}[\text{\Cast{\Ccode{t}}{[]}{\(\Ccode{t}_\Ccode{2}\)}}])
%\end{aligned}
%\]
%
%\[
%\begin{aligned}
%  \textit{iterate} : \text{CANON} + (\text{EVALCTX} \times \text{TERM}) &\to \text{CANON} +
%    (\text{EVALCTX} \times \text{TERM})\\
%  \textit{iterate}(\Ccode{v}) &= \Ccode{v}\\
%  \textit{iterate}(\mathcal{E}, \Ccode{t})
%        &= \textit{refocus}(\Ccode{t}\Ccmath{'}, \mathcal{E})
%    &\text{when } \Ccode{t}\rightsquigarrow\Ccode{t}\Ccmath{'}\\
%\end{aligned}
%\]
%
%\[
%\begin{aligned}
%  \Rfa : \text{EVALCTX} \times \text{CANON} &\to \text{CANON} +
%    (\text{EVALCTX} \times \text{TERM})\\
%  \Rfa([], \Ccode{v}) &= \textit{iterate}(\Ccode{v})\\
%  \Rfa(\mathcal{E}[[]\ \Ccode{t}], \Ccode{v}) &= \textit{iterate}(\mathcal{E}, \Ccode{v t}) \\
%  \Rfa(\mathcal{E}[\Ccode{ind}_\Ccode{I}\Ccode{(} []
%          \Ccode{, z.P, f.}\overline{\overline{\Ccode{y}}\Ccode{.t}}\Ccode{)}], \Ccode{v})
%        &= \textit{iterate}(\mathcal{E}, \Ccode{ind}_\Ccode{I}\Ccode{(v, z.P, f.}
%            \overline{\overline{\Ccode{y}}\Ccode{.t}}\Ccode{)}) \\
%  \Rfa(\mathcal{E}[\Ccode{?}_{[]}], \Ccmath{\Pi}\Ccode{(x:A).B})
%        &= \textit{iterate}(\mathcal{E}, \Ccode{?}_{\Ccmath{\Pi}\Ccode{(x:A).B}}) \\
%  \Rfa(\mathcal{E}[\Ccode{err}_{[]}], \Ccmath{\Pi}\Ccode{(x:A).B})
%        &= \textit{iterate}(\mathcal{E}, \Ccode{err}_{\Ccmath{\Pi}\Ccode{(x:A).B}}) \\
%  \Rfa(\mathcal{E}[\Ccode{?}_{[]}], \Ccode{v})
%        &= \Rfa{}(\mathcal{E}, \Ccode{?}_\Ccode{v}) \\
%  \Rfa(\mathcal{E}[\Ccode{err}_{[]}], \Ccode{v})
%        &= \Rfa{}(\mathcal{E}, \Ccode{err}_\Ccode{v}) \\
%  \Rfa(\mathcal{E}[\text{\Cast{\Ccode{t}}{[]}{\Ccode{t}\(\Ccmath{'}\)}}], \Ccode{?}_{\Ccmath{\square}})
%        &= \textit{refocus}(\Ccode{t},
%            \mathcal{E}[\text{\Cast{[]}{\Ccode{?}\(_{\Ccmath{\square}}\)}{\Ccode{t}\(\Ccmath{'}\)}}]) \\
%  \Rfa(\mathcal{E}[\text{\Cast{\Ccode{t}}{[]}{\Ccode{t}\(\Ccmath{'}\)}}], \Ccode{v})
%        &= \textit{refocus}(\Ccode{t}\Ccmath{'}, \mathcal{E}[\text{\Cast{\Ccode{t}}{\Ccode{v}}{[]}}])
%    & \text{when \Ccode{v}} \in
%        \{\Ccmath{\square},\ \Ccmath{\Pi}\Ccode{(x:A).B},\ \Ccode{I(}\overline{\Ccode{a}}\Ccode{)} \}\\
%  \Rfa(\mathcal{E}[\text{\Cast{\Ccode{t}}{[]}{\Ccode{t}\(\Ccmath{'}\)}}], \Ccode{v})
%        &= \textit{iterate}(\mathcal{E}, \text{\Cast{\Ccode{t}}{\Ccode{v}}{\Ccode{t}\(\Ccmath{'}\)}}) \\
%  \Rfa(\mathcal{E}[\text{\Cast{\Ccode{t}}{\Ccode{t}\(\Ccmath{'}\)}{[]}}], \Ccode{v})
%        &= \textit{refocus}(\Ccode{t}, \mathcal{E}[\text{\Cast{[]}{\Ccode{t}\(\Ccmath{'}\)}{\Ccode{v}}}])
%    & \text{when \Ccode{v}} \in
%        \{\Ccmath{\Pi}\Ccode{(x:A).B},\ \Ccode{I(}\overline{\Ccode{a}}\Ccode{)} \}\\
%  \Rfa(\mathcal{E}[\text{\Cast{\Ccode{t}}{\Ccode{t}\(\Ccmath{'}\)}{[]}}], \Ccode{v})
%        &= \textit{iterate}(\mathcal{E}, \text{\Cast{\Ccode{t}}{\Ccode{t}\(\Ccmath{'}\)}{\Ccode{v}}}) \\
%  \Rfa(\mathcal{E}[\text{\Cast{[]}{\Ccode{t}}{\Ccode{t}\(\Ccmath{'}\)}}], \Ccode{v})
%        &= \textit{iterate}(\mathcal{E}, \text{\Cast{\Ccode{v}}{\Ccode{t}}{\Ccode{t}\(\Ccmath{'}\)}}) \\
%\end{aligned}
%\]
\bibliographystyle{plainnat} \bibliography{ref}

\end{document}
